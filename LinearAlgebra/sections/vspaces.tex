\subsection{Fields}

\begin{definition}
	A \emph{field} is a set $\FF$ together with two binary operations. Addition is a map:
	\begin{equation*}
		+ : \FF^2 \to \FF, \ (a,b) \mapsto a+b,$ 
	\end{equation*}
	such that, if $\alpha, \beta\in \FF$, then the following properties are satisfied:
\begin{enumerate}
	\item
		Addition is commutative, i.e., $\alpha + \beta = \beta + \alpha;$
	\item
		It is associative, i.e., if $\gamma \in \FF$ then $(\alpha + \beta) + \gamma = \alpha + (\beta + \gamma);$
	\item
		There exists an element (identity) $z \in \FF$, such that $\alpha + z = \alpha$. It will be shown that this element is unique, thus it will always be denoted by $0$ and called zero;
	\item
		Every element is invertible, that is, there exists $l$ such that $\alpha + l = 0$. As in the previous property, the additive inverse of an element $\alpha$ in uniquely determined, and thus will be denoted by $-\alpha$.
\end{enumerate}
Multiplication, frequently denoted by $\times$ or $\cdot$, is a map:
\begin{equation*}
	\cdot : \FF^2 \to \FF, \ (a,b) \mapsto a \cdot b;
\end{equation*}
satisfying, for all $\alpha, \beta \in \FF$:
\begin{enumerate}
	\item
		$\alpha \beta = \beta \alpha$;
	\item
		If $\gamma \in \FF$ then $(\alpha \beta) \gamma = \alpha (\beta \gamma)$;
	\item
		There exists $e \in \FF$ such that $e \alpha = \alpha$ for every $\alpha \in \FF$. This element is unique and denoted by $1$;
	\item
		For every $\alpha \neq 0$, there exists $\gamma \in \FF$ such that $\alpha \gamma = 1$. The element $\gamma$ is uniquely determined by $\alpha$ so it will be denoted by $\alpha^{-1}$.
	\item \emph{Multiplication is distributive over addition}, i.e., $\alpha( \beta + \gamma) = \alpha \beta + \alpha \gamma$.
\end{enumerate}

\end{definition}


\begin{example}
	
\begin{itemize}
	\item $\RR, \CC$ and $\QQ$ are the most commonly encountered fields.
	\item The set $ \QQ[\sqrt 2] = \{a + b \sqrt 2 : a, b \in \QQ \}$
\end{itemize}	
\end{example}

\begin{proposition}[Basic properties of fields]
	Let $\FF$ be a field.
\begin{enumerate}
	\item Both additive and multiplicative identities are unique; 
	\item $0 \neq 1$;
	\item For all $\alpha \in \FF$, $\alpha 0 = 0$.
\end{enumerate}
\end{proposition}
\begin{proof}
	\begin{enumerate}
	\item	
		Let $\alpha$ be an element of $\FF$. Let $e, e^\prime \in \FF$ both additive identities, then $e = e + e^\prime = e^\prime + e = e^\prime$. The uniqueness of the multiplicative identity follows from an identical argument.	
	\end{enumerate}
\end{proof}

