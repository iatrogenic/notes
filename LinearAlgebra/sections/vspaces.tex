\subsection{Fields (optional)}

\begin{definition}
	A \emph{field} is a set $\FF$ together with the two following binary operations. Addition is a map:
	\begin{equation*}
		+ : \FF \times \FF \to \FF; \quad (a,b) \mapsto a+b,
	\end{equation*}
	such that, if $\alpha, \beta\in \FF$, then the following properties are satisfied:
\begin{enumerate}
	\item
		Addition is commutative, i.e., $\alpha + \beta = \beta + \alpha;$
	\item
		It is associative, i.e., if $\gamma \in \FF$ then $(\alpha + \beta) + \gamma = \alpha + (\beta + \gamma);$
	\item
		There exists an element (called \emph{additive identity}) $z \in \FF$, such that $\alpha + z = \alpha$. It will be shown that this element is unique, thus it will always be denoted by $0$ and called zero;
	\item
		Every element is invertible, that is, there exists $l$ such that $\alpha + l = 0$. As in the previous property, the additive inverse of an element $\alpha$ in uniquely determined, and thus will be denoted by $-\alpha$.
\end{enumerate}
Multiplication, frequently denoted by $\times$ or $\cdot$, is a map:
\begin{equation*}
	\cdot : \FF \times \FF \to \FF; \quad (a,b) \mapsto a \cdot b = ab,
\end{equation*}
satisfying, for all $\alpha, \beta \in \FF$:
\begin{enumerate}
	\item
		$\alpha \beta = \beta \alpha$;
	\item
		If $\gamma \in \FF$ then $(\alpha \beta) \gamma = \alpha (\beta \gamma)$;
	\item
		There exists $e \in \FF$ (called an \emph{multiplicative identity}) such that $e \neq 0$ and $e \alpha = \alpha$ for every $\alpha \in \FF$. This element is unique and denoted by $1$;
	\item
		For every $\alpha \neq 0$, there exists $\gamma \in \FF$ such that $\alpha \gamma = 1$. The element $\gamma$ is uniquely determined by $\alpha$ so it will be denoted by $\alpha^{-1}$.
	\item \emph{Multiplication is distributive over addition}, i.e., $\alpha( \beta + \gamma) = \alpha \beta + \alpha \gamma$.
\end{enumerate}

\end{definition}


\begin{example}
	$\RR, \CC$ and $\QQ$ are the most commonly encountered fields.
\end{example}

\begin{example}
	The set $ \QQ[\sqrt 2] = \{a + b \sqrt 2 : a, b \in \QQ \}$ together with the usual addition and multiplication is a field.
\end{example}

\begin{proposition}
	Let $\FF$ be a field.
\begin{enumerate}
	\item Both additive and multiplicative identities are unique; 
	\item For all $\alpha \in \FF$, $\alpha 0 = 0$;
	\item For all $\alpha \in \FF$, $- \alpha = (-1) \alpha$.
\end{enumerate}
\end{proposition}
\begin{proof}
	\textbf{1.):}
	Let $e, e^\prime \in \FF$ both additive identities, then $e = e + e^\prime = e^\prime + e = e^\prime$. The uniqueness of the multiplicative identity follows from an identical argument.	
\par \textbf{2.)} 
Let $\alpha \in \FF$, $\alpha 0 = \alpha (0 + 0) = \alpha 0 + \alpha 0$, add the additive inverse of $\alpha 0$ to both sides and we get the desired equality. \textbf{3.)}  $\alpha + (-1)\alpha = (1 + (-1))\alpha = 0\alpha$ which is equal to $0$ by the previous proposition. Thus we get $\alpha + (-1)\alpha = \alpha - \alpha \Leftrightarrow (-1) \alpha = - \alpha$. 
\end{proof}

\subsection{Vector Spaces}
\begin{definition}
	Let $V$ be a set and $\FF$ an arbitrary field, provided with with the mappings:
	\begin{itemize}
		\item
			$(\alpha, \beta) \mapsto \alpha + \beta; \quad V \times V \to V,$ called \emph{addition};
		\item
			$(x, \alpha) \mapsto x\alpha; \quad \FF \times V \to V$, called \emph{scalar multiplication}.
	\end{itemize}
	$V$ is said to be a \emph{vector space over $\FF$} with respect to these operations if:
	\begin{enumerate}
		\item
			For all $\alpha, \beta, \gamma \in V$ the equation $\alpha + (\beta + \gamma) = (\alpha + \beta) + \gamma$ holds;
		\item
			For all $\alpha, \beta \in V$, $\alpha + \beta = \beta + \alpha$;
		\item
			There exists $0 \in V$ such that $\alpha + 0 = \alpha$ for all $\alpha \in V$;
		\item
			For every $\alpha \in V$, there exists $\beta \in V$ such that $\alpha + \beta = 0$;
		\item
			Let $1 \in \FF$ be its multiplicative identity, then $1 \alpha = \alpha$ for all $\alpha \in V$;
		\item
			For all $x,y \in \FF$ and $\alpha, \beta \in V$, $x(\alpha + \beta) = x\alpha + x\beta$ and $(x+y) \alpha = x \alpha + y \alpha$.
	\end{enumerate}
	If $\FF = \RR$ then $V$ is said to be a \emph{real vector space}. If $\FF = \CC$ then $V$ is said to be a \emph{complex vector space}. 
\end{definition}

\begin{example}
	Consider the set $\CC$ of complex numbers. This set, together with a scalar multiplication map $(r, z) \in \RR \times \CC \mapsto rz \in \CC$ and the usual complex number addition, form a real vector space.
\end{example}

\begin{notation}
	Let $F$ and $A$ be sets. By $F^A$ we mean the set of functions from $A$ to $F$. Consider $\RR^\RR$ as an example, this is the set of all real-valued functions of one real variable.
\end{notation}

\begin{example}
	The set $\RR^A$, where $A$ is an arbitrary set, under the natural operations of addition of two functions and multiplication of a function by a real number, is a real vector space. Note that $\RR^n$ with $n \in \NN$ is a particular example of such sets, since $\RR^{ \{ 1, \ldots, n \} } = \RR^n$. \footnote{To be more precise, they are isomorphic. This concept will be explored later.}	
\end{example}

\begin{proposition}
Let $V$ be a vector space over $\FF$, then:
\begin{enumerate}
	\item
		The additive identity, denoted $0$, is unique;
	\item Let $\alpha \in V$ then its additive inverse, denoted $-\alpha$, is unique;
	\item
		$\forall \alpha \in V : 0 \alpha = 0$;
	\item
		$\forall x \in \FF : x0 = 0$;
	\item
		$\forall \alpha \in V : (-1) \alpha = - \alpha$.
\end{enumerate}
\end{proposition}

\subsubsection{Subspaces}
\begin{definition}
	Let $V$ be a vector space over $\FF$. A subset $U$ of $V$ is said to be a \emph{subspace} of $V$ if it too is a vector space over $\FF$.
\end{definition}

\begin{proposition}
	Let $V$ be a vector space over $\FF$ and $U \subseteq V$. $U$ is a subspace of $V$ if and only if $U$ satisfies the following conditions:
	\begin{enumerate}
		\item
			$0 \in U$;
		\item
			$\alpha, \beta \in U \Rightarrow \alpha + \beta \in U$;
		\item
			$x \in \FF$ and $\alpha \in U \Rightarrow x \alpha \in U$.
	\end{enumerate}
\end{proposition}

\begin{definition} 
	Let $U_1, \ldots, U_m$ be subsets of $V$. We define its \emph{sum} as
	\begin{equation*}
		U_1 + \cdots + U_m \coloneqq \{ u_1 + \cdots + u_m : u_1 \in U_1 , \ldots, u_m \in U_m \}. 
	\end{equation*}
\end{definition}
