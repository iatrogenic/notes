
\documentclass[12pt]{article}

\usepackage[utf8]{inputenc}
\usepackage{mathtools}
\usepackage{bm}
\usepackage{amsfonts}
\usepackage{enumitem}
\usepackage{amssymb}
\usepackage{amsthm}
\usepackage{fullpage}
\usepackage{commath}

\newcommand{\CC}{\mathbb{C}}
\newcommand{\KK}{\mathbb{K}}
\newcommand{\PP}{\mathbb{P}}
\newcommand{\RR}{\mathbb{R}}
\newcommand{\NN}{\mathbb{N}}
\newcommand{\QQ}{\mathbb{Q}}
\newcommand{\ZZ}{\mathbb{Z}}
\newcommand{\pset}{\mathcal{P}}

\newtheorem{theorem}{Theorem}[section]
\newtheorem{corollary}{Corollary}[theorem]
\newtheorem{lemma}[theorem]{Lemma}
\newtheorem{prop}{Proposition}[theorem]

\theoremstyle{definition}
\newtheorem{example}{Example}
\newtheorem{problem}{Problem}
%\newtheorem*{solution}{Solution}
\newtheorem{definition}{Definition}[section]
\newtheorem{remark}{Remark}
\newtheorem*{notation}{Notation}

\newenvironment{solution}{\paragraph{Solution:}}{\hfill$\blacksquare \\$}

\DeclareMathOperator{\Ima}{Im}

\title{Notes on Complex Analysis}
\author{}
\begin{document}

\maketitle
\tableofcontents
\newpage
\section{Power Series in One Variable}
\subsection{The Field of Complex Numbers}

\subsection{Formal Power Series}
Let $\KK$ be a field. The set of formal polynomial in indeterminate $X$ with coefficients in $\KK$ is denoted by $\KK[X]$. This set, equipped with the usual polynomial addition and scalar multiplication, is an infinite dimensional vector space, spanned by the basis: \[ 1, X, \ldots, X^n, \ldots \]

	Multiplication of polynomials defines a bilinear product. This additional operation turns $\KK[X]$ into an \emph{algebra}. If we drop the ``finitely many non-zero coefficients" requirement in the definition of polynomial we arrive at the definition of a \emph{formal power series}. The set of all formal power series with coefficients in field $\KK$ is denoted by $\KK[[X]]$. The algebra $\KK[X]$ is identified with a subalgebra of $\KK[[X]]$.
\par
Let $S = \sum_{n \geq 0} a_nX^n \neq 0$ be a formal power series. The \emph{order} of $S$, denoted by $\omega(S)$, is the smallest $k$ such that $a_k \neq 0$. The order of $0 \in \KK[[X]]$ is $\infty$.
\par
A family of power series $(S_i \in \KK[[X]]: i \in I)$  is said to be \emph{summable} if for any $k \in \NN$, $\omega(S_i) \geq k$ for all but a finite number number of indices $I$. The \emph{sum} of a summable family offormal series
\[
	S_i(X) = \sum_{n \geq 0} a_{n,i} X^n
\]
is the series
\[ 
S(X) = \sum_{n \geq 0} a_n X^n
\]
where, for each n, $a_n = \sum_i a_{n,i}$.

\end{document}
