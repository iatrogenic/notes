\documentclass[]{article}
%chalk
\newcommand{\PP}{\mathbb{P}}
\newcommand{\RR}{\mathbb{R}}
\newcommand{\NN}{\mathbb{N}}
\newcommand{\QQ}{\mathbb{Q}}
\newcommand{\ZZ}{\mathbb{Z}}

%other
\newcommand{\pset}{\mathcal{P}}

%packages
\usepackage{amsmath}
\usepackage{amsfonts}
\usepackage{amsthm}
\usepackage{enumitem}
\usepackage{amssymb}
%math environments
\newtheorem{theorem}{Theorem}[section]
\newtheorem{corollary}{Corollary}[theorem]
\newtheorem{lemma}[theorem]{Lemma}

\theoremstyle{definition}
\newtheorem{problem}{Problem}
%\newtheorem*{solution}{Solution}
\newtheorem{definition}{Definition}[section]
\newtheorem{remark}{Remark}
\newtheorem{notation}{Notation}

\newenvironment{solution}{\paragraph{Solution:}}{\hfill$\blacksquare \\$}


%opening
\title{Advanced Calculus}
\author{}
\date{}
\begin{document}

\maketitle
\newpage
\tableofcontents
\newpage

\section{Vector Spaces}
\subsection{Fundamental Notions}
\subsubsection{Summary}
Not much is covered here. Vector spaces and subspaces are defined.
The following theorem is not mentioned (so far) in the book but it is quite useful for checking whether or not a certain set is a subspace.
\begin{theorem}\label{subspace:1}
	A subset $U$ of $V$ is a subspace of $V$ if and only if $U$ satisfies the following conditions:
	\begin{enumerate}
		\item $0 \in U$,
		\item $\alpha, \beta \in U$ implies $\alpha + \beta \in U$,
		\item $x \in \RR$ and $\alpha \in U$ implies $x\alpha \in U$.
	\end{enumerate}
\end{theorem}
\subsubsection{First set of exercises}
In what follows, A1, A2, A3, A4, S1, S2, S3, S4 refer to the axioms presented in the book.
\begin{problem}
	Prove S3 for $\RR^3$ using the explicit display form $\{x_1, x_2, x_3\}$ for ordered triples.
\end{problem}
\begin{solution}

With $(x_1, x_2, x_3), (y_1, y_2, y_3) \in \RR^3$ and $x \in \RR$. 
	\begin{align*} 
		x((x_1, x_2, x_3) + (y_1, y_2, y_3)) &= x(x_1 + y_1, x_2 + y_2, x_3 + y_3) 8 \\ 
	 &=  (x(x_1 + y_1), x(x_2 + y_2), x(x_3 + y_3)) \\
	 &= (xx_1 + xy_1, xx_2 + xy_2, xx_3 + xy_3) \\
	 &= (xx_1, xx_2, xx_3) + (xy_1, xy_2, xy_3) \\
	 &= x(x_1, x_2, x_3) + x(y_1, y_2, y_3).
	\end{align*}
\end{solution}
\begin{problem}
	Show that given $\alpha$, the $\beta$ postualted in A4 is unique.
\end{problem}
\begin{solution}
	Let $\alpha, \beta, \beta^\prime \in V$ a vector space over $\RR$, such that $\alpha + \beta = 0$ and $\alpha + \beta^\prime = 0$. By transitivity
	\begin{align*}
		\alpha + \beta &= \alpha + \beta^\prime \\
		(\beta + \alpha) + \beta &= (\beta + \alpha) + \beta^\prime \\
		0 + \beta &= 0 + \beta^\prime \\
		\beta &= \beta^\prime
	\end{align*}
\end{solution}
\begin{problem}
	Prove similarly that $0\alpha = 0$, $x0 = 0$, and $(-1)\alpha = -\alpha$.
\end{problem}
\begin{solution}
For the first equality, 
\begin{align*}
	0\alpha &= (0 + 0)\alpha \\
	& = 0\alpha + 0\alpha,
\end{align*}
 by S2. Subtracting $0\alpha$ from both sides yields the desired identity. Proving $x0 = 0$ is identical, except S3 is used instead of S2. As for the last equation:
 \begin{align*}
 	\alpha + (-1)\alpha &= (1 - 1)\alpha \\
 	&= 0\alpha \\
 	&= 0.
 \end{align*}
 We already determined that $-\alpha$ is unique, so it follows that $(-1)\alpha = -\alpha$.
\end{solution}
\begin{problem}
	Prove that if $x\alpha = 0$, then either $x=0$ or $\alpha=0$.
\end{problem}
\begin{solution}
	Assume $a \neq 0$ and $x \neq 0$. Since $\alpha \in \RR$ has an inverse $\alpha^{-1}$.
	\begin{align*}
		x\alpha = 0 &= 0 \\
		x\alpha \alpha^{-1} &= 0 \alpha^{-1} \\
		x &= 0.
	\end{align*}
	Contradiction.
\end{solution}
\begin{problem}
	Prove S1 for a function space $\RR^4$. Prove S3.
\end{problem}
\begin{solution}
	Let $x,y$ be elements of $\RR$, $f$ and $g$ real functions on $A$, and $a$ an element of $A$. \\
	Since $f(a)$ is a real number, $(xy)f(a) = x(yf(a))$ is just a consequence of associativity in $\RR$. As for S3, note also that $g(a) \in \RR$, thus
	\begin{align*}
		x(f + g)(a) &= x(f(a) + g(a)) \\
		&= xf(a) + xg(a).
	\end{align*}
Because no conditions were imposed on $a$, both equalities are valid for all elements of $A$.
\end{solution}

\begin{problem}
	Given that $\alpha$ is any vector in a vector space $V$, show that the set $A = \{x\alpha \mid x \in \RR \}$ of all scalar multiples of $\alpha$ is a subspace of $V$.
\end{problem}
\begin{solution}
	We can see that $0 \in A$  because $0 \alpha = 0$. Now take $\beta$ and $\gamma$ elements of $A$, 
	\begin{align*}
		\beta + \gamma &= x \alpha + y \alpha \\ 
		&= (x + y)\alpha,
	\end{align*}
	which is clearly an element of $A$. Finally, taking $y \in \RR$, $y(x\alpha) = (yx) \alpha \in A$.
	By theorem \ref{subspace:1}, $A$ is a subspace of $V$.
\end{solution}
\begin{problem}
	Given that $\alpha$ and $\beta$ are any two vectors in $V$, show that the set of all vectors $x\alpha + y\beta$, where $x$ and $y$ are any real numbers, is a subspace of $V$.
\end{problem}
\begin{solution}
	Setting $x = y = 0$ shows that the additive identity is in the set. Let $\gamma = x\alpha + y\beta$ and $\delta = x'\alpha + y'\beta$.
	\begin{align*}
		\gamma + \delta &= (x\alpha + y\beta) + x'\alpha + y'\beta \\
						&= (x+x')\alpha + (y + y')\beta.
	\end{align*}
	Finally, if $z \in \RR$ then $z(x\alpha + y\beta) = (zx)\alpha + (zy)\beta$.
\end{solution}
\begin{problem}
	Show that the set of triples $\mathbf{x}$ in $\RR^3$ such that $x_1 - x_2 + 2x_3 = 0$ is a subspace $M$. If $N$ is the similar subspace $\{ \mathbf{x} \mid x_1 + x_2 + x_3 = 0 \}$, find a nonzero vector $\textbf{a}$ in $M \cap N$. Show that $M \cap N$ is the set $\{x\mathbf{a} \mid x \in \RR \}$ of all scalar multiples of $\mathbf{a}$.
\end{problem}
\begin{solution}
The intersection $M \cap N$ is the set of all triples $\mathbf{x} = (x_1, x_2, x_3)$ that satisfy the system:
\[
\begin{cases}
	x_1 - x_2 + 2x_3 = 0 \\
	x_1 + x_2 + x_3 = 0
\end{cases}
\]
The nonzero triple $\mathbf{a} = (3,-1,-2)$ satisfies this system. \\ 
Let $ A = \{x\mathbf{a} \mid x \in \RR\}$. Clearly, if $x \in \RR$, then $x\mathbf{a} \in M \cap N$, that is, $A \subset M \cap N$.
Now let $\alpha = (a_1, a_2, a_3)$ be an element of $M \cap N$, then
\begin{align*}
a_1 - a_2 + 2a_3 &= a_1 + a_2 + a_3	\\
-2a_2 + a_3 &= 0 \\
a_3 &= 2a_2, 
\end{align*}
and now substituting $a_3$ by $2a_2$,
\begin{align*}
a_1 + a_2 + 2a_2 &= 0 \\
a_1 &= -3a_2.	
\end{align*}
So $\alpha = (a_1, a_2, a_3) = (a_1, - \frac{1}{3}a_1, -\frac{2}{3}a_1) = a_1(1, - \frac{1}{3}, - \frac{2}{3}) = a_13\mathbf{a}$

\end{solution}
\begin{problem}
Let $A$ be the open interval $(0,1)$, and let $V$ be $\RR^A$.
Given a point $x$ in $(0,1)$, let $V_x$ be the set of functions in $V$ that have a derivative at $x$.
Show that $V_x$ is a subspace of $V$.
\end{problem}
\begin{solution}
The constant function $I(x) = 0$ has a derivative at $x$ and it is the identity of $\RR^A$. Given $f(x),g(x)$	
functions in $V_x$, we know that $(f+g)'(x) = f'(x) + g'(x) \Rightarrow f+g \in V_x$. Also, given $y \in \RR$, $(yf(x))' = yf'(x) \Rightarrow yf(x) \in V_x$.
\end{solution}
\begin{problem}
 For any subsets $A$ and $B$ of a vectors space $V$ we define the set sum $A+B$ by $A+B = \{\alpha + \beta \mid \alpha \in A \text{ and } \beta \in B\}$.
 Show that $(A+B)+C = A+(B+C)$.	
\end{problem}
\begin{solution}
\begin{align*}
(A+B)+C &= \{ (\alpha + \beta) + \gamma \mid \alpha \in A, \beta \in B, \gamma \in C\} \\
&= \{ \alpha + (\beta + \gamma) \mid \alpha \in A, \beta \in B, \gamma \in C\} \\
&= A + (B+C).	
\end{align*}
\end{solution}
\begin{problem}
Let $A \subset V$ and $X \subset \RR$, we similarly define, $XA = \{x\alpha \mid x \in X \text{ and } \alpha \in A\}$. 
Show that a nonvoid set $A$ is a subspace if and only if $A+A = A$ and $\RR A = A$.	
\end{problem}
\begin{solution}
	Let $A \neq \emptyset$. \\
($\Rightarrow$) Suppose $A$ is a subspace, clearly $A+A = A$ by closure of addition on vector spaces. Likewise, $\RR A = A$ by closure under scalar multiplication. \\
($ \Leftarrow $) Suppose $A+A=A$ and $\RR A = A$. Since $A \neq \emptyset$ then $0 \in A$. Let $\alpha, \beta$ be elements of $A$, $\alpha + \beta \in A+A = A$ by hypothesis. Finally, let $x$ be a real number and $\alpha$ and element of $A$, then $x\alpha \in \RR A = A$. By \ref{subspace:1}, A is a subspace of $V$.
\end{solution}
\begin{problem}
Let $V$ be $\RR^2$, and let $M$ be the line through the origin with slope $k$. Let $\mathbf{x}$ be any nonzero vector in $M$. Show that $M$ is the subspace $\RR\mathbf{x} = \{t\mathbf{x} \mid t \in \RR\}$.	
\end{problem}
\begin{solution}
$M$ is the set $\{(x, kx) \in \RR^2\} \subset \RR^2$. Let $\mathbf{x} = (x_1, kx_1) \neq 0$, if $t \in \RR$ then $t\mathbf{x} = (tx_1, tkx_1) \in M$, hence $\RR\mathbf{x} \subset M$.
Conversely, if $(y, ky) \in M$ and $(y, ky) = y(1, k) = y\frac{1}{x_1}(x_1, kx_1) = yx_1^{-1}\mathbf{x} \in \RR\mathbf{x} \Rightarrow M \subset \RR\mathbf{x}$. Thus $M = \RR\mathbf{x}$.
\end{solution}
\begin{problem}
Show that any other line $L$ with the same slope $k$ is of the form $M + \mathbf{a}$ for some $\mathbf{a}$.
\end{problem}
\begin{solution}
Let $L$ be a line with slope $k$
\begin{align*}
	L &= \{(x, kx+b) \mid x \in \RR \} \\
	  &= \{(x,kx) + (0,b) \mid x \in \RR \} \\
	  &= \{(x, kx) \mid x \in \RR\} + (0,b) \\
	  &= M + (0,b).
\end{align*}	

\end{solution}
\begin{problem}
	Let $M$ be a subspace of a vector space $V$, and let $\alpha$ and $\beta$ be any two vectors in $V$.
	Given $A = \alpha + M$ and $B = \beta + M$, show that either $A = B$ or $A \cap B = \emptyset$.
	Show also that $A+B=(\alpha + \beta)  + M$.	
\end{problem}
\begin{solution}
	
	For the second proposition:
	\begin{align*}
		A + B &= (\alpha + M) + (\beta + M) \\
			  &= \{x + y \mid x \in \alpha + M \text{ and } y \in \beta + M \} \\
			  &= \{\alpha + \xi + \beta + \zeta \mid \xi, \zeta \in M\} \\
			  &= (\alpha + \beta) + M.
	\end{align*}
\end{solution}
\begin{problem}
State more carefully and prove what is meant by ``a subspace of a subspace is a subspace''.	
\end{problem}
\begin{solution}
	Let $V$ be a vector space and $A$ a subspace of it. If $B$ is a subspace of $A$ then $B$ is also a subspace of $V$.
	\\
	To prove this, note that $B \subset A \subset V \Rightarrow B \subset V$, by hypothesis, all of the propositions of theorem \ref{subspace:1} are true and thus $B$ is a subspace of $V$.
\end{solution}
\begin{problem}
Prove that the intersection of two subspaces of a vector space is always itself a subspace.	
\end{problem}
\begin{solution}
Let $V$ be a vector space and $A,B$ subspaces of $V$. $0 \in A$ and $0 \in B \Rightarrow 0 \in A \cap B$. If $\alpha,\beta \in A \cap B$ then $\alpha,\beta \in A$ and $\alpha, \beta \in B$, since both are subspaces, $\alpha+\beta \in A$ and $\alpha + \beta \in B \Rightarrow \alpha+\beta \in A \cap B$. If $x \in \RR$ and $\alpha \in A \cap B$.
\begin{align*}
\alpha \in A \cap B & \Rightarrow \alpha \in A \wedge \alpha \in B \\
& \Rightarrow x\alpha \in A \wedge x\alpha \in B \\
& \Rightarrow x\alpha \in A \cap B.
\end{align*} 
\end{solution}
\begin{problem}
Prove more generally that the intersection $W = \cap_{i\in I} W_i$ of any family $\{W_i \mid i \in I \}$ of subspaces of $V$ is a subspace of $V$.
\end{problem}
\begin{solution}
	By definiton of subspace, $0$ must be an element of every $W_i$ and is consequently an element of $W$. To check closure of addition consider $\alpha, \beta$ elements of $W$,
	\begin{align*}
	\alpha, \beta \in W & \Rightarrow \alpha, \beta \in W_i & \forall i \in I \\
	& \Rightarrow \alpha + \beta \in W_i & \forall i \in I \\
	& \Rightarrow \alpha + \beta \in W.
	\end{align*}
	Closure under scalar multiplication follows by similar reasoning. 
\end{solution}
\begin{problem}
Let $V$ again be $\RR^{(0,1)}$, and let $W$ be the set of all functions $f$ in $V$ such that $f'(x)$ exists for every $x$ in $(0,1)$. Show that $W$ is the intersection of the collection of subspaces of the form $V_x$ that were considered in problem 9.	
\end{problem}
\begin{solution}
	Let $f$ be an element of $W$, clearly, \[f \in \bigcap_{x \in (0,1)} V_x \], which implies $W \subset \bigcap V_x$. Now suppose $f \in \bigcap_{x\in(0,1)} V_x$, then
	\begin{align*}
		f \in V_x & \Rightarrow \exists f'(x)  & \forall x \in (0,1) \\
		& \Rightarrow f \in W.
	\end{align*}
	Hence we obtain the desired equality.
\end{solution}
\begin{problem}
	Let $V$ be a function space $R^A$, and for a point $a$ in $A$ let $W_a$ be the set of functions such that $f(a) = 0$. $W_a$ is clearly a subspace. For a subset $B \subset A$ let $W_B$ be the set of functions $f$ in $V$ such that $f = 0$ in $B$. Show that $W_B$ is the intersection $\bigcap_{a \in B} W_a$.
\end{problem}
\begin{solution}
	Suppose $f \in W_B$, then
	\begin{align*}
		f(b) = 0 &\Rightarrow f \in W_b & \forall b \in B \\
		& \Rightarrow f \in \bigcap_{a \in B} W_a. & \\
		& \Rightarrow W_B \subset \bigcap_{a \in B} W_a. &
	\end{align*}
Now if $f \in \bigcap_{a \in B} W_a$, assume $f \notin W_B$, then there exists $b$ in $B$ such that $f(b) \neq 0$, but this would entail $f \notin W_b \Rightarrow f \notin \bigcap_{a \in B} W_a$, contradicting our hypothesis.
\end{solution}
\begin{problem}
	Supposing again that $X$ and $Y$ are subspaces of $V$, show that if $X+Y = V$ and $X \cap Y = \{0\}$, then for every vector $\zeta$ in $V$ there is a unique pair of vectors $\xi \in X$ and $\eta \in Y$ such that $\zeta = \xi + \eta$.
\end{problem}
\begin{solution}
	Let $\xi_1$ and  $\eta_1$ be elements of $X$ and $Y$ respectively, such that, $\xi + \eta = \zeta = \xi_1 + \eta_1$, so we have that 	\[
	\eta - \eta_1 = \xi_1 - \xi.
	\]
	Closure of addition in a vector space implies that $\eta_1 - \eta$ is an element of both $Y$ and $X$, likewise for $\xi$ and $\xi_1$, it follows that 
	\[
	\xi - \xi_1 = 0  = \eta - \eta_1.
	\]
	Thus $\xi = \xi_1$ and $\eta = \eta_1$.
\end{solution}
\begin{problem}
	Show that if $X$ and $Y$ are subspaces of a vector space $V$, then the union $X \cup Y$ can only be a subspace if either $X \subset Y$ or $Y \subset X$.
\end{problem}
\begin{solution}
	Let $X \cup Y$ be a subspace of $V$ and $\alpha \in X \cup Y$. By definition, $\alpha$ is an element of $X$ or an element of $Y$. For purposes of contradiction, assume $X$ is not a subset of $Y$ and vice versa. This implies that there exists $x \in X$ such that $x \notin Y$ and $y \in Y$ such that $y \notin X$. Since $X \cup Y$ is a subspace, $x+y$ is an element of it, call it $z$. Then $x = z -y$ and $y = z - x$. Since $z$ is an element of $X \cap Y$ it is an element of $X$ or an element of $Y$. If $z \in X$ then $y \in X$, by closure of addition in a subspace, this is a contradiction. If $z \in Y$ then we obtain $x \in Y$, which is also a contradiction.
\end{solution}
\subsubsection{Second set of exercises}
\begin{problem} Given  $ \alpha = (1,1,1) $,  $ \beta = (0,1,-1) $  ,  $ \gamma = (2,0,1) $, compute the linear combination  $ \alpha + \beta + \gamma, 3\alpha - 2\beta + \gamma, x\alpha + z\gamma $. Find  $ x,y $, and  $ z $  such that  $ x\alpha + y\beta + z\gamma = (0,0,1) = \delta^3$. Do the same for  $ \delta^1 $  and  $ \delta^2 $.      
\end{problem}
\end{document}
