\documentclass[]{article}
%chalk
\newcommand{\PP}{\mathbb{P}}
\newcommand{\RR}{\mathbb{R}}
\newcommand{\NN}{\mathbb{N}}
\newcommand{\QQ}{\mathbb{Q}}
\newcommand{\ZZ}{\mathbb{Z}}

%other
\newcommand{\pset}{\mathcal{P}}

%packages
\usepackage{amsmath}
\usepackage{amsfonts}
\usepackage{amsthm}
\usepackage{enumitem}
\usepackage{amssymb}
%math environments
\newtheorem{theorem}{Theorem}[section]
\newtheorem{corollary}{Corollary}[theorem]
\newtheorem{lemma}[theorem]{Lemma}

\theoremstyle{definition}
\newtheorem{problem}{Problem}
%\newtheorem*{solution}{Solution}
\newtheorem{definition}{Definition}[section]
\newtheorem{remark}{Remark}
\newtheorem{notation}{Notation}

\newenvironment{solution}{\paragraph{Solution:}}{\hfill$\blacksquare \\$}


%opening
\title{Advanced Calculus}
\author{}
\date{}
\begin{document}

\maketitle
\newpage
\tableofcontents
\newpage

\section{Vector Spaces}
\subsection{Fundamental Notions}
\subsubsection{First set of exercises}
In what follows, A1, A2, A3, A4, S1, S2, S3, S4 refer to the axioms presented in the book.
\begin{problem}
	Prove S3 for $\RR^3$ using the explicit display form $\{x_1, x_2, x_3\}$ for ordered triples.
\end{problem}
\begin{solution}

With $(x_1, x_2, x_3), (y_1, y_2, y_3) \in \RR^3$ and $x \in \RR$. 
	\begin{align*} 
		x((x_1, x_2, x_3) + (y_1, y_2, y_3)) &= x(x_1 + y_1, x_2 + y_2, x_3 + y_3) 8 \\ 
	 &=  (x(x_1 + y_1), x(x_2 + y_2), x(x_3 + y_3)) \\
	 &= (xx_1 + xy_1, xx_2 + xy_2, xx_3 + xy_3) \\
	 &= (xx_1, xx_2, xx_3) + (xy_1, xy_2, xy_3) \\
	 &= x(x_1, x_2, x_3) + x(y_1, y_2, y_3).
	\end{align*}
\end{solution}
\begin{problem}
	Show that given $\alpha$, the $\beta$ postualted in A4 is unique.
\end{problem}
\begin{solution}
	Let $\alpha, \beta, \beta^\prime \in V$ a vector space over $\RR$, such that $\alpha + \beta = 0$ and $\alpha + \beta^\prime = 0$. By transitivity
	\begin{align*}
		\alpha + \beta &= \alpha + \beta^\prime \\
		(\beta + \alpha) + \beta &= (\beta + \alpha) + \beta^\prime \\
		0 + \beta &= 0 + \beta^\prime \\
		\beta &= \beta^\prime
	\end{align*}
\end{solution}
\begin{problem}
	Prove similarly that $0\alpha = 0$, $x0 = 0$, and $(-1)\alpha = -\alpha$.
\end{problem}
\begin{solution}
For the first equality, 
\begin{align*}
	0\alpha &= (0 + 0)\alpha \\
	& = 0\alpha + 0\alpha,
\end{align*}
 by S2. Subtracting $0\alpha$ from both sides yields the desired identity. Proving $x0 = 0$ is identical, except S3 is used instead of S2. As for the last equation:
 \begin{align*}
 	\alpha + (-1)\alpha &= (1 - 1)\alpha \\
 	&= 0\alpha \\
 	&= 0.
 \end{align*}
 We already determined that $-\alpha$ is unique, so it follows that $(-1)\alpha = -\alpha$.
\end{solution}
\begin{problem}
	Prove that if $x\alpha = 0$, then either $x=0$ or $\alpha=0$.
\end{problem}
\begin{solution}
	Assume $a \neq 0$ and $x \neq 0$. Since $\alpha \in \RR$ has an inverse $\alpha^{-1}$.
	\begin{align*}
		x\alpha = 0 &= 0 \\
		x\alpha \alpha^{-1} &= 0 \alpha^{-1} \\
		x &= 0.
	\end{align*}
	Contradiction.
\end{solution}
\begin{problem}
	Prove S1 for a function space $\RR^4$. Prove S3.
\end{problem}
\begin{solution}
	Let $x,y$ be elements of $\RR$, $f$ and $g$ real functions on $A$, and $a$ an element of $A$. \\
	Since $f(a)$ is a real number, $(xy)f(a) = x(yf(a))$ is just a consequence of associativity in $\RR$. As for S3, note also that $g(a) \in \RR$, thus
	\begin{align*}
		x(f + g)(a) &= x(f(a) + g(a)) \\
		&= xf(a) + xg(a).
	\end{align*}
Because no conditions were imposed on $a$, both equalities are valid for all elements of $A$.
\end{solution}

The following theorem is not mentioned (so far) in the book but it is quite useful for checking whether or not a certain set is a subspace.
\begin{theorem}\label{subspace:1}
	A subset $U$ of $V$ is a subspace of $V$ if and only if $U$ satisfies the following conditions:
	\begin{enumerate}
		\item $0 \in U$,
		\item $\alpha, \beta \in U$ implies $\alpha + \beta \in U$,
		\item $x \in \RR$ and $\alpha \in U$ implies $x\alpha \in U$.
	\end{enumerate}
\end{theorem}
\begin{problem}
	Given that $\alpha$ is any vector in a vector space $V$, show that the set $A = \{x\alpha \mid x \in \RR \}$ of all scalar multiples of $\alpha$ is a subspace of $V$.
\end{problem}
\begin{solution}
	We can see that $0 \in A$  because $0 \alpha = 0$. Now take $\beta$ and $\gamma$ elements of $A$, 
	\begin{align*}
		\beta + \gamma &= x \alpha + y \alpha \\ 
		&= (x + y)\alpha,
	\end{align*}
	which is clearly an element of $A$. Finally, taking $y \in \RR$, $y(x\alpha) = (yx) \alpha \in A$.
	By theorem \ref{subspace:1}, $A$ is a subspace of $V$.
\end{solution}
\begin{problem}
	Given that $\alpha$ and $\beta$ are any two vectors in $V$, show that the set of all vectors $x\alpha + y\beta$, where $x$ and $y$ are any real numbers, is a subspace of $V$.
\end{problem}
\begin{solution}
	Setting $x = y = 0$ shows that the additive identity is in the set. Let $\gamma = x\alpha + y\beta$ and $\delta = x'\alpha + y'\beta$.
	\begin{align*}
		\gamma + \delta &= (x\alpha + y\beta) + x'\alpha + y'\beta \\
						&= (x+x')\alpha + (y + y')\beta.
	\end{align*}
	Finally, if $z \in \RR$ then $z(x\alpha + y\beta) = (zx)\alpha + (zy)\beta$.
\end{solution}
\begin{problem}
	Show that the set of triples $\mathbf{x}$ in $\RR^3$ such that $x_1 - x_2 + 2x_3 = 0$ is a subspace $M$. If $N$ is the similar subspace $\{ \mathbf{x} \mid x_1 + x_2 + x_3 = 0 \}$, find a nonzero vector $\textbf{a}$ in $M \cap N$. Show that $M \cap N$ is the set $\{x\mathbf{a} \mid x \in \RR \}$ of all scalar multiples of $\mathbf{a}$.
\end{problem}
\begin{solution}
The intersection $M \cap N$ is the set of all triples $\mathbf{x} = (x_1, x_2, x_3)$ that satisfy the system:
\[
\begin{cases}
	x_1 - x_2 + 2x_3 = 0 \\
	x_1 + x_2 + x_3 = 0
\end{cases}
\]
The nonzero triple $\mathbf{a} = (3,-1,-2)$ satisfies this system. \\ 
Let $ A = \{x\mathbf{a} \mid x \in \RR\}$. Clearly, if $x \in \RR$, then $x\mathbf{a} \in M \cap N$, that is, $A \subset M \cap N$.
Now let $\alpha = (a_1, a_2, a_3)$ be an element of $M \cap N$, then
\begin{align*}
a_1 - a_2 + 2a_3 &= a_1 + a_2 + a_3	\\
-2a_2 + a_3 &= 0 \\
a_3 &= 2a_2, 
\end{align*}
and now substituting $a_3$ by $2a_2$,
\begin{align*}
a_1 + a_2 + 2a_2 &= 0 \\
a_1 &= -3a_2.	
\end{align*}
So $\alpha = (a_1, a_2, a_3) = (a_1, - \frac{1}{3}a_1, -\frac{2}{3}a_1) = a_1(1, - \frac{1}{3}, - \frac{2}{3}) = a_13\mathbf{a}$

\end{solution}
\begin{problem}
Let $A$ be the open interval $(0,1)$, and let $V$ be $\RR^A$.
Given a point $x$ in $(0,1)$, let $V_x$ be the set of functions in $V$ that have a derivative at $x$.
Show that $V_x$ is a subspace of $V$.
\end{problem}
\begin{solution}
The constant function $I(x) = 0$ has a derivative at $x$ and it is the identity of $\RR^A$. Given $f(x),g(x)$	
functions in $V_x$, we know that $(f+g)'(x) = f'(x) + g'(x) \Rightarrow f+g \in V_x$. Also, given $y \in \RR$, $(yf(x))' = yf'(x) \Rightarrow yf(x) \in V_x$.
\end{solution}
\begin{problem}
 For any subsets $A$ and $B$ of a vectors space $V$ we define the set sum $A+B$ by $A+B = \{\alpha + \beta \mid \alpha \in A \text{ and } \beta \in B\}$.
 Show that $(A+B)+C = A+(B+C)$.	
\end{problem}
\begin{solution}
\begin{align*}
(A+B)+C &= \{ (\alpha + \beta) + \gamma \mid \alpha \in A, \beta \in B, \gamma \in C\} \\
&= \{ \alpha + (\beta + \gamma) \mid \alpha \in A, \beta \in B, \gamma \in C\} \\
&= A + (B+C).	
\end{align*}
\end{solution}
\begin{problem}
Let $A \subset V$ and $X \subset \RR$, we similarly define, $XA = \{x\alpha \mid x \in X \text{ and } \alpha \in A\}$. 
Show that a nonvoid set $A$ is a subspace if and only if $A+A = A$ and $\RR A = A$.	
\end{problem}
\begin{solution}
	Let $A \neq \emptyset$. \\
($\Rightarrow$) Suppose $A$ is a subspace, clearly $A+A = A$ by closure of addition on vector spaces. Likewise, $\RR A = A$ by closure under scalar multiplication. \\
($ \Leftarrow $) Suppose $A+A=A$ and $\RR A = A$. Since $A \neq \emptyset$ then $0 \in A$. Let $\alpha, \beta$ be elements of $A$, $\alpha + \beta \in A+A = A$ by hypothesis. Finally, let $x$ be a real number and $\alpha$ and element of $A$, then $x\alpha \in \RR A = A$. By \ref{subspace:1}, A is a subspace of $V$.
\end{solution}
\begin{problem}
Let $V$ be $\RR^2$, and let $M$ be the line through the origin with slope $k$. Let $\mathbf{x}$ be any nonzero vector in $M$. Show that $M$ is the subspace $\RR\mathbf{x} = \{t\mathbf{x} \mid t \in \RR\}$.	
\end{problem}
\begin{solution}
$M$ is the set $\{(x, kx) \in \RR^2\} \subset \RR^2$. Let $\mathbf{x} = (x_1, kx_1) \neq 0$, if $t \in \RR$ then $t\mathbf{x} = (tx_1, tkx_1) \in M$, hence $\RR\mathbf{x} \subset M$.
Conversely, if $(y, ky) \in M$ and $(y, ky) = y(1, k) = y\frac{1}{x_1}(x_1, kx_1) = yx_1^{-1}\mathbf{x} \in \RR\mathbf{x} \Rightarrow M \subset \RR\mathbf{x}$. Thus $M = \RR\mathbf{x}$.
\end{solution}
\begin{problem}
Show that any other line $L$ with the same slope $k$ is of the form $M + \mathbf{a}$ for some $\mathbf{a}$.
\end{problem}
\begin{solution}
	
\end{solution}
\end{document}
