\subsection{Basic Concepts}
\subsubsection*{Partial Functions}
A \emph{partial function} generalizes the usual definition of function, the idea being that this kind of function is potentially not defined on the entire domain. Formally:
\begin{definition}
	A \emph{partial funciton} $f$ from $X$ to $Y$ (written as $f : X \nrightarrow Y$)  is a triple $(g,X,Y)$ such that $X^\prime \subseteq X$ and $g:X^\prime \to Y$ is a function. Furthermore:
	\begin{itemize}
	 \item 
		The \emph{domain} of $f$ is denoted by $\Dom(f)$ and is equal to $X^\prime$; 
	 \item
		 If $ \Dom(f) = X$ then $f$ is a \emph{total function}\footnote{Total functions and functions are equivalent.};
	 \item
		 If $x \in (X \setminus \Dom f)$ then $f(x)$ is said to be \emph{undefined}, denoted $f(x) = \--$, on the other hand, if $x \in \Dom f$ then we write $f(x) = y$ with $y = g(x)$ and say that $f$ is \emph{defined} at $x$.  
 \end{itemize}
\end{definition}
Henceforth the word ``function" will always mean ``partial function."  As an example, consider the (partial) function:

\begin{align*}
	f : \NN_0  & \nrightarrow  \NN_0 \\
		n  & \mapsto       \sqrt n.
\end{align*}
If $n \in \NN_0$ is not a perfect square, then $f(n)$ is undefined. 

\subsubsection*{Lambda Notation}

\subsection{What is a computable function?}
\subsubsection*{Informal Discussion}
An \emph{algorithm} is a finite sequence of discrete mechanical instructions. A numerical function is \emph{effectively computable} (or simply \emph{computable}) if an algorithm exists that can be used to calculate the value of the function for any given input from its domain.

\subsubsection*{The Unlimited Register Machine}
The \emph{unlimited register machine} has an infinite number of \emph{registers} labelled $R_1, R_2, \ldots$, each containing a natural number, if $R_i$ is a register then $r_i$ is the number it contains. It can be represented as follows

\begin{center}
	\begin{tabular}{|c|c|c|c|c|c|c|c}
		\hline
		$R_1$ & $R_2$ & $R_3$ & $R_4$ & $R_5$ & $R_6$ & $R_7$ & $\cdots$\\ 
		\hline
		$r_1$ & $r_2$ & $r_3$ & $r_4$ & $r_5$ & $r_6$ & $r_7$ & $\cdots$  \\
		\hline
	\end{tabular}
\end{center}
The contents of a register may be altered by the URM in response to certain \emph{instructions}. 

\subsubsection*{Instruction set}
\begin{center}
	\begin{tabular}{|c|c|p{11cm}|}
	\hline 
	Name of  Instruction & Instruction & URM response \\
	\hline
	Zero & $\Iz{n}$ & $r_n \gets 0$ \\ 
	\hline
	Successor & $\Is{n}$ & $r_n \gets r_n + 1$ \\
	\hline
	Transfer & $\It{m}{n}$ & $r_n \gets r_m $ \\
	\hline
	Jump & $\Ij{m}{n}{q}$ & if $r_m = r_n$ then jump to $q$-th instruction; otherwise proceed to next instruction. \\
	\hline
\end{tabular}
\end{center}


