\subsection{Basic Concepts}
\subsubsection*{Partial Functions}
A \emph{partial function} generalizes the usual definition of function, the idea being that this kind of function is potentially not defined on the entire domain. Formally:
\begin{definition}
	A \emph{partial function} $f$ from $X$ to $Y$ (written as $f : X \nrightarrow Y$)  is a triple $(g,X,Y)$ such that $X^\prime \subseteq X$ and $g:X^\prime \to Y$ is a function. Furthermore:
	\begin{itemize}
	 \item 
		The \emph{domain} of $f$ is denoted by $\Dom(f)$ and is equal to $X^\prime$; 
	 \item
		 If $ \Dom(f) = X$ then $f$ is a \emph{total function}\footnote{Total functions and usual functions are equivalent.};
	 \item
		 If $x \in (X \setminus \Dom f)$ then $f(x)$ is said to be \emph{undefined}, denoted $f(x) = \--$, on the other hand, if $x \in \Dom f$ then we write $f(x) = y$ with $y = g(x)$ and say that $f$ is \emph{defined} at $x$.  
 \end{itemize}
\end{definition}
Henceforth the word ``function" will always mean ``partial function."  As an example, consider the (partial) function:

\begin{align*}
	f : \NN_0  & \nrightarrow  \NN_0 \\
		n  & \mapsto       \sqrt n.
\end{align*}
If $n \in \NN_0$ is not a perfect square, then $f(n)$ is undefined. 

\subsubsection*{Lambda Notation}
We will often use Alonzo Church's \emph{lambda notation}. Given a mathematical expression $a(x_1, \ldots, x_n)$ the function $f: \NN_0^n \to \NN_0$ that maps $(x_1, \ldots, x_n) \mapsto a(x_1, \ldots, x_n)$ may be denoted by $f = \lambda_{x_1, \ldots, x_n} \cdot a(x_1, \ldots, x_n)$ .
\subsection{The URM}
\subsubsection*{Informal Discussion}
An \emph{algorithm} is a finite sequence of discrete mechanical instructions. A numerical function is \emph{effectively computable} (or simply \emph{computable}) if an algorithm exists that can be used to calculate the value of the function for any given input from its domain.

\subsubsection*{The Unlimited Register Machine}
The \emph{unlimited register machine} has an infinite number of \emph{registers} labelled $R_1, R_2, \ldots$, each containing a natural number, if $R_i$ is a register then $r_i$ is the number it contains. It can be represented as follows

\begin{center}
	\begin{tabular}{|c|c|c|c|c|c|c|c}
		\hline
		$R_1$ & $R_2$ & $R_3$ & $R_4$ & $R_5$ & $R_6$ & $R_7$ & $\cdots$\\ 
		\hline
		$r_1$ & $r_2$ & $r_3$ & $r_4$ & $r_5$ & $r_6$ & $r_7$ & $\cdots$  \\
		\hline
	\end{tabular}
\end{center}
The contents of the registers determine its \emph{state} or \emph{configuration}, which might be altered by the URM in response to certain \emph{instructions}. 

\subsubsection*{URM Programs}
\begin{center}
	\begin{tabular}{|c|c|p{11cm}|}
	\hline 
	Name of  Instruction & Instruction & URM response \\
	\hline
	Zero & $\Iz{n}$ & $r_n \gets 0$ \\ 
	\hline
	Successor & $\Is{n}$ & $r_n \gets r_n + 1$ \\
	\hline
	Transfer & $\It{m}{n}$ & $r_n \gets r_m $ \\
	\hline
	Jump & $\Ij{m}{n}{q}$ & if $r_m = r_n$ then jump to $q$-th instruction; otherwise proceed to next instruction. \\
	\hline
\end{tabular}
\end{center}
Without exception, the parameters of these instructions are elements of $\NN_1$.
\begin{definition}
	An \emph{URM program} is a finite sequence of URM instructions.	The number of instructions of a program is denoted by $\#P$.
\end{definition}
Given a program $P = (I_1 , \ldots, I_n)$ the URM always starts by executing $I_1$, the execution flow then proceeds incrementally unless a jump instruction is performed. The machine's response to each instruction is described in the table above.

%initial configuration

\begin{definition}
	An URM program $P$ \emph{computes} the function $f : \NN_0^n \nrightarrow \NN_0$ if for every $a_1, \ldots, a_n, b \in \NN_0$ then:
	\begin{equation*}
		P(a_1, \ldots, a_n) \downarrow b \Leftrightarrow (a_1, \ldots, a_n) \in \Dom f \land f(a_1, \ldots, a_n) = b,	
	\end{equation*}
	and
	\begin{equation*}
		P(a_1, \ldots, a_n) \uparrow \, \Leftrightarrow f(a_1, \ldots, a_n) = \--	
	\end{equation*}
\end{definition}
The class of URM-computable functions is denoted by $\mathcal{C}$ and by $\mathcal{C}_n$ the class of $n$-ary computable functions. 

\begin{definition}
	Let $P$ be an URM program and $n \in \NN_1$. The unique $n$-ary function that $P$ computes is denoted by $f_P^{(n)}$, which, given any $x_1, \ldots, x_n \in \NN_0$, is defined by:
	\[
		f_P^{(n)}(x_1, \ldots, x_n) ) = 
		\left\{
	\begin{array}{ll}
		\-- & \mbox{if } P(x_1, \ldots, x_n) \uparrow \\
		y & \mbox{if } P(x_1, \ldots, x_n) \downarrow y 
	\end{array}
	\right. .
	\]
\end{definition}
\begin{example}
	Let $Q = (\Iz{2}, \Ij{1}{2}{6}, \Is{2}, \Ij{1}{1}{2}, \Is{3}, \It{2}{1})$. The binary function computed by $Q$ is $f_Q^{(2)}(x,y) = x$.
\end{example}
\begin{definition}
Let $M(x_1, \ldots, x_n)$ be an $n$-ary predicate. The characteristic function of the predicate $M$ is the function $C_M : \NN_0^n \to \NN_0$ defined by:
\[
	C_M(x_1, \ldots, x_n) = 
		\left\{
	\begin{array}{ll}
		1 & \mbox{if } M(x_1, \ldots, x_n) \\
		0 & \mbox{if } \neg M(x_1, \ldots, x_n) 
	\end{array}
	\right. .
\]
The predicate $M$ is \emph{decidable} if its characteristic function is computable and \emph{undecidable} when it is not.
\end{definition}

\subsubsection*{Building programs out of other programs}
\begin{definition}
	A program $P$ is in \emph{standard form} if, for every jump instruction $\Ij{m}{n}{q} \in P$ it holds that $q \leq \#P + 1$.
\end{definition}

\begin{definition}
	Let $P = (I_1, \ldots, I_n)$ be an URM program. We denote by $P^*$ the program $(I^\prime_1, \ldots, I^\prime_n)$ constructed as follows:
	\[
	I_i^\prime = 
		\left\{
	\begin{array}{ll}
		\Ij{m}{n}{\#P+1} & \mbox{if } I_i = \Ij{m}{n}{k}, \text{ with } k > \#P+1 \\
		I_i & \mbox{otherwise} 
	\end{array}
	\right. .
	\]
\end{definition}

\begin{definition}
	Two programs $P_1$ and $P_2$ are \emph{(strongly) equivalent} if, for any initial configuration $(a_1, a_2, a_3, \ldots)$ it holds that:
	\begin{itemize}
		\item $P_1(a_1, a_2, a_3, \ldots) \downarrow$ iff $P_2(a_1, a_2, a_3, \ldots) \downarrow$;
		\item when it is the case that both computations $P_1(a_1, a_2, a_3, \ldots)$ and $P_2(a_1, a_2, a_3, \ldots)$ stop, the final configurations of both machines are equal. 
	\end{itemize}
\end{definition}

\begin{theorem}
	Let $P$ be a URM program. The program $P^*$ is in standard form and is equivalent to $P$. 
\end{theorem}

\begin{definition}
	Let $P$ and $Q$ be URM programs. The \emph{concatenation} of $P$ and $Q$, denoted by $P;Q$, is the URM program defined as follows:
	\begin{itemize}
		\item $\#(P;Q) = \# P + \# Q$,
		\item For every $l \in \{ 1 , \ldots, \# P \}, (P;Q)[l] = P^\prime[l]$,
		\item For every $k \in \{ 1 , \ldots, \# Q \}$:
			\[
				(P;Q)[\#P+k] = 
		\left\{
	\begin{array}{ll}
		Q[k] & \mbox{if }  Q[k] \text{ is not a jump instruction}\\
		\Ij{m}{n}{r+ \#P} & \mbox{ if } Q[k] = \Ij{m}{n}{r} 
	\end{array}
	\right. .
			\]
	\end{itemize}	
\end{definition}

\begin{definition}
	Let $P$ be an URM program. If $ \{ R_{v_1}, \ldots, R_{v_n} \} $ is the set of registers mentioned in program $P$, we denote by $\rho(P)$ the number $ \max \{ v_1, \ldots, v_n \} $. 
\end{definition}

\begin{definition}
	Let $n, j \geq 1$ and $i_1, \ldots, i_n > n$ and $P$ be an URM program. We denote by $P[i_1, \ldots, i_n \to j]$ the following URM program:
	\[
		(\It{i_1}{1}, \ldots, \It{i_n}{n}, \Iz{n+1}, \ldots, \Iz{\rho(P)}); P ; (\It{1}{j}),
	\]
	where the sequence of instructions $\Iz{n+1}, \ldots, \Iz{\rho(P)}$ only occurs if $\rho(P) > n$.
\end{definition}

\begin{definition}
	A \emph{generalized URM program} $Q$ is a finite sequence of \emph{generalized URM instructions} $(I_1, \ldots, I_k)$, with $k \geq 1$, where each each of this sequence's elements is either a standard URM instruction or one of the following two (called $P$-calling instructions):
	\begin{enumerate}
		\item $\Ic{P}$,
		\item $\Icall{P}{i_1, \ldots, i_n}{j}$,
	\end{enumerate}
	where $n,j \geq 1$ and $i_1, \ldots, i_n > n$ and $P$ is an URM program which does not contain instructions that call program $Q$ or instructions that call other programs that call program $Q$.
\end{definition}
\subsection{Primitive and Partial Recursive Functions}
\begin{definition}
	The following functions are called \emph{basic functions}:
	\begin{enumerate}
		\item The \emph{zero} functions: $\mathsf{zero} = \lambda_x \cdot 0$,
		\item the \emph{successor} function: $\mathsf{suc} = \lambda_x \cdot x + 1$,
		\item for each $n \in \NN_1$ and $i \in \{ 1, \ldots, n \}$, the \emph{projection} function: $U_i^n = \lambda_{x_1, \ldots, x_n} \cdot x_i$.
	\end{enumerate}
\end{definition}

\begin{theorem}
	The basic functions are URM-computable.
\end{theorem}

\begin{definition}
	Let $f : \NN_0^k \nrightarrow \NN_0$ and $g_i : \NN_0^n \nrightarrow \NN_0$, for each $i \in \{ 1, \ldots, k \}$. Define $g = (g_1, \ldots, g_k)$ as  
	\begin{align*}
	g : \NN_0^n  & \nrightarrow  \NN_0^k \\
	(x_1, \ldots, x_n)  & \mapsto  g(x_1, \ldots, x_n) \\ 
	& \simeq (g_1(x_1, \ldots, x_n), \ldots, g_k(x_1, \ldots, x_n)).
	\end{align*}
	The composition of $f$ and $g = (g_1, \ldots, g_k)$, denoted $f \circ g$, is the following function:
	\begin{align*}
		f \circ g : \NN_0^n & \nrightarrow \NN_0 \\
		(x_1, \ldots, x_n) & \mapsto (f \circ g)(x_1, \ldots, x_n) \simeq f(g_(x_1, \ldots, x_n)) \\
				   & \simeq f(g_1(x_1, \ldots, x_n), \ldots, g_k(x_1, \ldots, x_n)).
	\end{align*}
Both of these functions are defined at a point $(x_1, \ldots, x_n)$ if and only if $g_1, \ldots, g_n$ are defined at $(x_1, \ldots, x_n)$.
\end{definition}

\begin{theorem}
	Let $f:\NN_0^k \nrightarrow \NN_0$ and $g_i : \NN_0^n \nrightarrow \NN_0$, for each $i \in \{ 1, \ldots, k \}$, are computable functions, then the function $h = f \circ ( g_1, \ldots, g_k) : \NN_0^n \nrightarrow \NN0$ defined, for any $(x_1, \ldots, x_n) \in \NN_0^n$, by $h(x_1, \ldots, x_n) \simeq f(g_1(x_1, \ldots, x_n), \ldots, g_k(x_1, \ldots, x_n))$, is also computable. In other words, composition preserves computability.
\end{theorem}

\begin{definition}
	Let $k \in \NN_0$ and $n \in \NN_1$. We denote by $\mathsf{k}^{(n)}$ the $n$-ary constant function define as follows:
	\begin{align*}
		\mathsf{k}^{(n)} : \NN_0^n & \nrightarrow \NN_0 \\
		(x_1, \ldots, x_n) & \mapsto \mathsf{k}^{(n)}(x_1, \ldots, x_n) = k
	\end{align*}
\end{definition}

\begin{theorem}
	For every $k \in \NN_0$ and every $n \in \NN_1$ the function $\mathsf{k}^{(n)}$ is obtained by composition of basic functions.
\end{theorem}

% TO-FINISH THIS

\subsection{Bounded and Unbounded Quantification}

\subsection{Alternative Models of Computability}
\subsubsection*{Turing Machines}
% TO-DO Introduce a visual representation of Turing Machines
% This is pretty bad

A \emph{Turing Machine} $M$ is an abstract device which ``performs" operations on a tape of infinite length in both directions. This tape is divided in individual squares along its length. At any given moment each square contains a single symbol from a fixed and finite set called the \emph{alphabet of} $M = \{ s_0, \ldots, s_n \}$. We assume that $s_0$ is the \emph{blank} symbol $\beta$ used to denote an empty square. The machine $M$ has a \emph{reading head} which, at any given moment, scans and reads a single square of the tape. This machine is capable of performing the following three kinds of operations on the tape:
\begin{enumerate}
\item Erase the symbol of the square being scanned and write one of the symbols in the alphabet;
\item Move the reading head one square to the right of the square being scanned;
\item Move the reading head one square to the left of the square being scanned.
\end{enumerate}
At any given moment, the machine $M$ is in one of a finite number of \emph{states}, represented by $q_1, q_2, \ldots, q_m$. The execution of an operation may cause the state of $M$ to change.

% Picture here

The operation to be performed by $M$ is determined by its \emph{specification}, denoted by $Q$, and its current state. The set $Q$ is finite and its elements are quadruples, each of which is of the form $(q_i, s_j, s_k, q_l)$, or $(q_i, s_j, R, q_l)$, or $(q_i, s_l, L, q_l)$, with $i,l \in \{ 1, \ldots, m \}$ and $j,k \in \{ 1, \ldots, m \}$. \par
A quadruple of the form $(q_i, s_j, \alpha, q_l)$ (where $\alpha \in \{ s_k , R, L \}$ in $Q$ specifies the action to be perform by $M$ when it is in state $q_i$ and reading the symbol $s_j$, as follows:
\begin{enumerate}
	\item Execute the following operation on the tape:
		\subitem If $\alpha = s_k$, erase $s_j$ and write $s_k$ in the square being scanned;
		\subitem If $\alpha = R$, move the reading head one square to the right;
		\subitem If $\alpha = L$, move the reading head one square to the left.
	\item Change into state $q_l$.
\end{enumerate}
In order for a machine $M$ to begin a computation an initial state is required and its reading head must be positioned over a single square of a given tape. Then $M$ starts performing the actions determined by its specification, as described above. The computation terminates if $M$ is in a state $q_p$ and reading a symbol $s_r$ such that there is no quadruple in $Q$ which starts with $q_p \, s_r$. Unless stated otherwise, given a Turing machine $M$ and an infinite tape, $M$ starts its computation in the state $q_1$ and with its reading head placed over the leftmost non-blank square of the received tape.

\begin{example}
	Let $M$ be a Turing machine with alphabet $\{ \beta, a , b \}$. This machine may be in only one of two states $q_1$ and $q_2$. Let $Q$ be the following specification of $M$:
	\begin{gather*}	
	q_1 \, a \, b \, q_2 \\
	q_1 \, b \, a \, q_2 \\
	q_2 \, a \, R \, q_1 \\
	q_2 \, b \, R \, q_1
	\end{gather*}
\end{example}

A numerical function $f: \NN_0^n \nrightarrow \NN_0$ is \emph{Turing-computable} if there exists a Turing machine that computes $f$. The class such functions is denoted by $\mathcal{T}$.


%To-Add Later
