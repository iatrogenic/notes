\subsubsection*{Basic Concepts}
A \emph{partial function} is a map $f$ denoted $f : X \nrightarrow Y$ such that there exists a subset $X' \subseteq X$ for which $f : X^\prime \to Y$ is a function. If $X = X^\prime$ then $f$ is said to be a \emph{total function}. The domain of $f$ is denoted by $\Dom f$ and defined by the set $\{ x : f(x) \text{ is defined} \}$; we say that $f(x)$ is undefined if $x \notin \Dom f$. The range of $f$, denoted by $\Ran f$ is the set $\{ f(x) : x \in \Dom f \}$. Partial functions generalize the concept of function which here corresponds to the definition of total function. 

\subsection{What is a computable function?}
Informally, an \emph{algorithm} is a sequence of mechanical instructions. A numerical function is \emph{effectively computable} if an algorithm exists that can be used to calculate the value of the function for any given input from its domain.
