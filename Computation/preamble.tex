\usepackage[utf8]{inputenc}
\usepackage{amsmath}
\usepackage{amsfonts}
\usepackage{enumitem}
\usepackage{amssymb}
\usepackage{amsthm}
\usepackage{fullpage}
\usepackage{mathtools}
\usepackage{blindtext}
\usepackage{subfiles}
\usepackage{titlesec}

\newcommand{\KK}{\mathbb{K}}
\newcommand{\FF}{\mathbb{F}}
\newcommand{\PP}{\mathbb{P}}
\newcommand{\RR}{\mathbb{R}}
\newcommand{\NN}{\mathbb{N}}
\newcommand{\CC}{\mathbb{C}}
\newcommand{\QQ}{\mathbb{Q}}
\newcommand{\ZZ}{\mathbb{Z}}
\newcommand{\pset}{\mathcal{P}}

\newtheorem{theorem}{Theorem}[section]       
\newtheorem{corollary}{Corollary}[theorem]
\newtheorem{lemma}[theorem]{Lemma}
\newtheorem{proposition}[theorem]{Proposition}
\theoremstyle{definition}
\newtheorem{problem}{Problem}
\newtheorem{definition}{Definition}[section]
\newtheorem{example}{Example}[section]
\newtheorem{remark}{Remark}
\newtheorem{notation}{Notation}

\newenvironment{solution}{\paragraph{Solution:}}

\titleformat{\subsubsection}[runin]{\bfseries}{1}{0em}{§ }[]

\DeclareMathOperator{\Dom}{\mathsf{Dom}}
\DeclareMathOperator{\Ran}{\mathsf{Ran}}
% 
\newcommand{\Iz}[1]{\texttt{Z(} #1 \texttt{)}}
\newcommand{\Is}[1]{\texttt{S(} #1 \texttt{)}}
\newcommand{\It}[2]{\texttt{T(} #1 \texttt{,} #2 \texttt{)} }
\newcommand{\Ij}[3]{\texttt{J(} #1\texttt{,} #2 \texttt{,} #3 \texttt{)} } 
\newcommand{\Icall}[3]{ \texttt{Call#1[} #2 \to #3 \texttt ] }
\newcommand{\Ic}[1]{ \texttt{Call#1}}

\title{Computation; Notes}
\author{}
