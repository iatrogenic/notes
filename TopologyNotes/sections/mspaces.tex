\subsection{Metrics}
\begin{definition}
	A function $d : E \times E \to \RR$ is said to be a \emph{metric} on a set $E$ if, for any $x,y,z \in E$:
\begin{enumerate}
	\item
		$d(x,y) \geq 0$;
	\item
		$d(x,y) = 0 \Leftrightarrow x = y$;
	\item
		$d(x,y) = d(y,x)$;
	\item
		$d(x,y) \leq d(x,z) + d(z,y)$.
\end{enumerate}
The pair $(E, d)$ is called a \emph{metric space} if $d$ is a metric on $E$.
\end{definition}

TO-DO: Insert Examples

\subsection{The Topology of a Metric Space}
Given any metric space $(E, d)$ it is possible to define a topology on it. Consider the sets
\begin{align*}
	B_{op}(a;R) \coloneqq \{ x \in E : d(x,a) < R \}, \\
	B_{cl}(a;R) \coloneqq \{ x \in E : d(x,a) \leq  R \}
\end{align*}
The former is called the \emph{open ball} (the latter, \emph{closed ball}) with center $a$ and radius $R$.
We also define the \emph{sphere} centered at $a$ with radius $R$ to be the set
\begin{equation*}
	S(a;R) \coloneqq \{ x \in E : d(x,a) = R \}.
\end{equation*}
By convention $B(a;R) = B_{op}(a;R)$. \par
The geometric motivation for these definitions becomes clear once the $\RR^2$ case is considered. 

TO-DO: Insert pictures

From these definitions we may easily deduce that:
\begin{align*}
	B_{cl}(a;R) = B_{op}(a;R) \cup S(a;R) ; \\
	B_{op}(a;R) \subset B_{cl}(a;R) ; \\
	R < R^\prime \Rightarrow B_{cl}(a;R) \subset B_{op}(a;R).
\end{align*}

\begin{definition}
	Let $(E, d)$ be a metric space and $X \subset E$. The set $X$ is said to be \emph{bounded} if $\exists a \in E$ and $\exists R \in \RR^+$, such that $X \subseteq B_{op}(a;R)$. \par
	Analogously, a function $f : I \to E$ (with $I$ an arbitrary set) is bounded if $f(I)$ is a bounded set. In particular, a sequence $(x_n)_{n \in \NN}$ on $E$ is bounded if:
	\begin{equation*}
		\exists a \in E, \exists R \in \RR^+ : \forall n \in \NN : d(x_n, a) \leq R .
	\end{equation*}
\end{definition}


\begin{definition}
	Let $(E,d)$ be a metric space. Consider the following set:
	\begin{equation*}
		\tau_d \coloneqq \{O \subset E : \forall a \in O, \exists R \in \RR^+ : B(a;R) \subset O \}.
	\end{equation*}
	This set will be called the \emph{generated topology by the metric} $d$ on $E$.
\end{definition}

\begin{theorem}
	Every metric space is Hausdorff.	
\end{theorem}

Apply concepts from topology to derive a "definition" of convergence for metric spaces.

\begin{definition}
	Let $d_1$ and $d_2$ be two metrics on $E$; they are said to be \emph{equivalent} if $\tau_{d_1} = \tau_{d_2}$.
\end{definition}

\begin{proposition}
	Let $d_1, d_2$ be metrics on $E$.
	If there exists $\alpha, \beta \in \RR^+$ such that, for every $x,y\in E$:
	\begin{equation*}
		\alpha d_1(x,y) \leq d_2(x,y) \leq \beta d_1(x,y)	
	\end{equation*}
	then, $d_1$ and $d_2$ are equivalent metrics.
\end{proposition}
\subsection{Continuity}
\begin{definition}
	Let $(E_1, \d_1)$ and $(E_2, d_2)$ be metric spaces. A function $f : E_1 \to E_2$ is \emph{continuous at point} $a \in E_1$ if:
	\begin{equation*}
		\forall \epsilon > 0, \exists \delta > 0, \forall x \in E_1 : x \in B(a; \delta) \Rightarrow f(x) \in B(f(a); \epsilon).
	\end{equation*}
	Two alternative equivalent statements:
	\begin{gather*}
		\forall \epsilon > 0, \exists \delta > 0, \forall x \in E_1: d_1(x;a) \leq \delta \Rightarrow d_2(f(x); f(a)) \leq \epsilon; \\ 
		\forall \epsilon > 0, \exists \delta > 0: f(B(a; \delta)) \subset B(f(a); \epsilon).
	\end{gather}
	The function $f$ is said to be \emph{continuous}, if $f$ is continuous at every point in $E_1$.
% motivate uniform continuity
	If $f$ is continuous and additionally satisfies:
	\begin{equation*}
		\forall \epsilon > 0, \exists \delta > 0, \forall x, a \in E_1 : d_1(x;a) \leq \delta \Rightarrow d_2(f(x);f(a)) \leq \epsilon
	\end{equation*}
	then $f$ is said to be \emph{uniformly continuous}. 
\end{definition}


\begin{definition}
	A function $f: E_1 \to E_2$ is \emph{Lipschitz continuous} if there exists $C \in \RR^+$, such that:
	\begin{equation*}
		\forall x,y \in E_1 : d_2(f(x), f(y)) \leq C d_1(x,y).
	\end{equation*}		
	To the real number $L_f$ defined by:
	\begin{equation*}
		L_f = \inf \{ C \in \RR : \forall x,y \in E_1 : d_2(f(x),f(y)) \leq C d_1(x,y) \},
	\end{equation*}
		is called a \emph{Lipschitz constant}.
\end{definition}



\begin{definition}
	function is a contraction
\end{definition}

\begin{theorem}
	Let $I$ be a non-degenerate interval of $\RR$ and let $f : I \to \RR$ be a continuous function on $I$ and differentiable on $\Int(I)$. Then $f$ is lipschitzian, if and only if, $f^\prime$ is bounded. In this case $L_f = \sup _{y \in \interior(I)]} \mid f^\prime (y) \mid$.
\end{theorem}
