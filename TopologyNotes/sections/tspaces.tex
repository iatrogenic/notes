\subsection{Topology}

\begin{definition}
    Let $E$ be a set. The set $\tau \subseteq \pset(E)$ is said to be a \emph{topology on $E$} if it satisfies the following axioms:
    \begin{enumerate}
        \item $\emptyset, E \in \tau$,
        \item $O_1, O_2 \in \tau \Rightarrow (O_1 \cap O_1 \in \tau)$,
        \item $(O_j \in \tau : j \in J) \Rightarrow (\cup_{j \in J} O_j) \in \tau$.
    \end{enumerate} 
    A \emph{topological space} is an ordered pair $(E, \tau)$ where $E$ is a set and $\tau$ a topology on $E$.
    An element of $X \in \pset(E)$ is said to be an \emph{open set} if $X \in \tau$, and said to be a \emph{closed set} if $E \setminus X \in \tau$.
\end{definition}

\begin{definition}
    A topological space $(E, \tau)$ is said to \emph{Hausdorff} if
    \begin{equation*}
        \forall a,b \in E \text{ such that } a \neq b, \text{ there exists } O_a, O_b \in \tau \text{ and } O_a \cap O_b \neq \emptyset
    \end{equation*}
\end{definition}

\begin{definition}
    Let $\tau_1, \tau_2$ be two topologies on a set $E$. The topology $\tau_1$ is said to be \emph{finer} than $\tau_2$ if $\tau_2 \subseteq \tau_1$.
    If in addition $\tau_1 \neq \tau_2$, then $\tau_1$ is said to be \emph{strictly finer} than $\tau_2$. 
\end{definition}

Two topologies are said to be \emph{comparable} if one is finer than the other.

\subsection{Interior, exterior, boundary and closure}

\begin{definition}
    Let $(E, \tau)$ be a topological space, $a$ an element of $E$, and $X \in \pset(E)$. Then:
    \begin{align*}
        a \text{ is an interior point of } X && \Leftrightarrow && \exists O \in \tau, a \in O \text{ and } 0 \subseteq X, \\
        a \text{ is an exterior point of } X && \Leftrightarrow && \exists O \in \tau, a \in O \text{ and } 0 \subseteq E \setminus X, \\
        a \text{ is a boundary point of } X && \Leftrightarrow && \forall O \in \tau, a \in O , O \cap X \neq \emptyset \text{ and } O \cap (E \setminus X) \\
    \end{align*}
\end{definition}

\begin{notation}
Note that the following sets depends on the fixed topology.
    \begin{itemize}
        \item $\interior X$ is the set of all interior points of $X$,
        \item $\exterior X$ is the set of all exterior points of $X$,
        \item $\partial X$ is the set of all boundary points of $X$,
        \item The set $\overline X = \interior X \cup \partial X$ is called the \emph{closure} of $X$.
    \end{itemize}
\end{notation}

\begin{remark}
	The elements of $\overline X$ are called \emph{adherent} point of $X$. It follows that
	\begin{equation*}
		a \text{ is an adherent point of } X \Leftrightarrow \forall O \in \tau, a \in O \Rightarrow (O \cap X) \neq \emptyset.
	\end{equation*}
\end{remark}

\begin{definition}
	Let $(E,\tau$ be a topological space, let $X$ be a subset of $E$ and $a \in E$. The element $a$ is said to be an \emph{accumulation} point of $X$ if:
	\begin{equation*}
	\forall O \in \tau, a \in O \Rightarrow \exists b, a \neq b.	
	\end{equation*}
	The set of such points is called the \emph{derivative} of $X$, denoted $X^\prime$. It follows that $X \cup X^\prime = \overline X$.
\end{definition}

\begin{proposition}
    Let $(E, \tau)$ be a topological space and $X \in \pset(E)$. Then
    \begin{enumerate}
        \item $a \in \overline{X} \Leftrightarrow \forall O \in \tau , a \in O \Rightarrow (O \cap X) \neq \emptyset$,
        \item $\interior X \subseteq X \subseteq \overline{X}$,
        \item $X \in \tau \Leftrightarrow X = \int X$,
        \item $(E \setminus X) \in \tau \Leftrightarrow \overline{X} = X$,
        \item $\overline{X} = X \cup \partial X$,
        \item $X \subseteq Y \Rightarrow \interior X \subseteq \interior Y$ and $\overline{X} \subseteq \overline{Y}$,
        \item $\overline{X \cup Y} = \overline{X} \cup \overline{Y}$ and $\overline{X \cap Y} \subseteq \overline{X} \cap \overline{Y}$,
        \item $\overline{\overline{X}} = X$.
    \end{enumerate}
\end{proposition}

\begin{definition}
	Let $(E,\tau)$ be a topological space, and $X, Y \subseteq E$. The set $X$ is said to be \emph{dense} on $Y$ if $Y \subseteq \overline{X}$ .
\end{definition}


\begin{definition}
	Let $(E, \tau)$ be a topological space; it is said to be \emph{separable} if there exists a subset $X$ of $E$ that is both dense and countable.
\end{definition}

\subsection{Neighborhoods}

\begin{definition}
	Let $(E, \tau)$ be a topological space and $a \in E$. A subset X of $E$ is said to be a \emph{neighborhood} of $a$ if there exists $O \in \tau$, such that $a \in O$ and $O \subseteq X$. \par
	The set $X$ is said to be a neighborhood of $A \subseteq E$, if for every element $a \in A$, X is a neighborhood of $a$. The set of all neighborhoods of a point $a$ is denoted by $\mathcal{V}_a$, and $\mathcal{V}_A$ denotes the set of all neighborhoods of an arbitrary set $A$. 
\end{definition}

\begin{proposition}
	\begin{gather*}
		X \in \tau \Leftrightarrow X \text{ is a neighborhood of } x \in X, \\
		V_1, V_2 \in \mathcal{V}_a \Rightarrow (V_1 \cap V_2) \in \mathcal{V}_a, \\
		V \in \mathcal{V}_a \text{ and } V \subseteq W \Rightarrow W \in \mathcal{V}_a, \\
		V \in \mathcal{V}_a \Rightarrow \interior(V) \in \mathcal{V}_a.
	\end{gather*}
\end{proposition}

\begin{definition}
	Let $\mathcal{W}$ be a class of neighborhoods of a given point $a$ (or a given set $X$) on the topological space $(E, \tau)$. We call $\mathcal{W}$ a \emph{fundamental neighborhood system} of $a$ (or $X$) if every neighborhood $V$ of $a$ (or $X$), is an element of $\mathcal{W}$, i.e., $V \in \mathcal{W}$.  
\end{definition}

\begin{definition}
	A topological space $(E, \tau)$ is said to satisfy the \emph{first axiom of numerability} if every $a \in E$, has a countable fundamental neighborhood system. 
\end{definition}

\begin{definition}
	Let $(J, \leq)$ be a well-ordered set. A class of neighborhoods $(V_j)_{j \in J}$ of $a$, indexed on J, is said to be a \emph{nested} fundamental system of neighborhoods of $a$, if $V_k \subseteq V_j$ whenever $j \leq k$. 
\end{definition}
